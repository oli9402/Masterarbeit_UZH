\section{Introduction} \label{Introduction} 
\subsection{Memory and Learning}
Memory and learning are concepts that are closely related. When it comes to defining these concepts, no clear consensus has emerged, but most people have an implicit understanding of these terms and know how to apply them \parencite[]{dehouwerWhatLearningNature2013, kleinWhatMemory2015}. Some definitions of memory and learning are kept as general as possible to ensure the inclusion of all different disciplines that investigate memory and learning \parencite{barronEmbracingMultipleDefinitions2015, kleinWhatMemory2015}. For example, a broad definition of memory is that it reflects organically based activity where information is made available through perception or introspection and thus alters the neural machinery \parencite{kleinWhatMemory2015}. A similar broad working definition of learning is that it reflects “processing of information derived from experience to update system properties” \parencite[page. 405]{barronEmbracingMultipleDefinitions2015}. Other definitions are more pragmatic and related to specific subdisciplines such as cognitive psychology \parencite[see,][]{barronEmbracingMultipleDefinitions2015}. For example, definitions found in cognitive psychology are that learning reflects the act of storing information in memory \parencite{medinCognitivePsychology1992}, or that memory is the product of learning and the process of retention and retrieval \parencite{crowderPrinciplesLearningMemory1976}. Difficulties in defining learning and memory in an all-inclusive way may reflect the complexity of these constructs and the limitations that still exist when it comes to understanding these concepts. However, with memory being a heavily studied construct in psychology\footnote{For example, Google Scholar search with ["memory"] yielded 1'720'000 results}, much insight has already been gained.  
For example, it is commonly accepted that new information passes through different stages and ends up as permanent memory in long-term memory \parencite{abenDistinctionWorkingMemory2012}. These stages are encoding (i.e., initial learning), storage (i.e., maintaining information), and retrieval (i.e., access stored information) \parencite{meltonImplicationsShorttermMemory1963}. 
Over the years, it became clear that memory is not a unitary system, which led to a taxonomy of memory systems with a main distinction between memory about facts and events (i.e., declarative) and unconscious memory such as, for example, skill and habit learning (i.e. non-declarative)\parencite{cohenPreservedLearningRetention1980, squireMechanismsMemory1986, squireMemorySystemsBrain2004}. What started out as a philosophical inquiry into memory through introspection \parencite{kahanaFoundationsHumanMemory2012} has led to an interdisciplinary field that incorporates research from psychology, neuropsychology, biology, and neuroscience \parencite{kahanaFoundationsHumanMemory2012, polsterCognitiveNeuroscienceAnalyses1991, squireMemoryBrainSystems2009}.
\subsubsection{Beginning of Experimental Study of Human Memory}
One of the first scientific endeavors in studying human memory came from Hermann Ebbinghaus in 1885 \parencite{ebbinghausUberGedachtnisUntersuchungen1885, kahanaFoundationsHumanMemory2012}. Arguing for experimental methods with highly controlled conditions, Ebbinghaus constructed learning material of syllables, which he used in experiments to study his own memory performance. Focusing on controlled experiments rather than studying memory in its complex manifestation in daily life, Ebbinghaus established the foundations for modern experimental study of memory \parencite{kahanaFoundationsHumanMemory2012}. Years later, after building on numerous experimental studies, a highly influential theory of memory (i.e., modal model by \textcite{atkinsonHumanMemoryProposed1968}) was postulated \parencite{malmberg50YearsResearch2019}. As described in \textcite{atkinsonHumanMemoryProposed1968} the modal model theory views memory as having three structural components: sensory register, short-term store (STS) and long-term store (LTS). The sensory register is where incoming stimuli are first stored, but decay after a period of several hundred milliseconds\footnote{In subsequent versions of the model, the sensory register is combined with the short-term store to form a single component \parencite{atkinsonControlShortTermMemory1971}}. The second component is the STS, where a limited amount of information can be stored. It was assumed that the stored information decays over a time period considerably longer than that of the sensory register. Finally, the long-term store is where unlimited information is stored relatively permanently. It was proposed that information could be transferred bidirectionally from one memory component to the next. Information transfer occurs through control processes. These processes can be described as strategies that individuals use to encode, store, and retrieve information from the different memory stores. Short-term store when cognitive processes are active is labeled working memory (WM) \parencite{atkinsonHumanMemoryProposed1968} and plays an important role in learning \parencite{baddeleyWorkingMemory2010}.

\subsubsection{Working Memory}
Following criticism \parencite[see.,][]{raaijmakersStoryTwostoreModel1993}, \textcite{baddeleyWorkingMemory1974} proposed a theory of working model to replace STS from the modal model. Their model is composed of two domain-specific stores: the phonological loop (verbal information) and the visuospatial sketch pad (visual-spatial information),  a central executive that coordinates the processes of the storage systems, and an episodic buffer, which holds semantic information associated with both domain-specific stores \parencite{baddeleyEpisodicBufferNew2000, baddeleyWorkingMemory1974}.  Through this model, the concept of working memory became famous \parencite{abenDistinctionWorkingMemory2012} and led to increased interest in the research of WM with different theories being proposed \parencite{cowanManyFacesWorking2017, oberauerBenchmarksModelsShortterm2018}. With the increase in research, the term WM has received different meanings depending on the theoretical framework in which it was used in \parencite[see.,][]{cowanManyFacesWorking2017}. Therefore, its distinction from concepts such as short-term memory became blurred and may even be seen as the same \parencite[see.,][]{abenDistinctionWorkingMemory2012}. However, most frameworks describe the basic feature of WM as a system for the limited short-term maintenance of information \parencite{erikssonNeurocognitiveArchitectureWorking2015} that can be accessed for a variety of abilities such as learning, reasoning, and preparation for action \parencite{baddeleyWorkingMemory1974}. It is suggested that a network of frontoparietal brain areas is activated during WM tasks, where the parietal regions are believed to be involved in manipulating and updating information in WM, and the frontal regions serve to monitor and organize memory processes \parencite{cohenTemporalDynamicsBrain1997, koenigsSuperiorParietalCortex2009,vanasselenBrainAreasInvolved2006}.

\subsubsection{Development of Working Memory and Its Importance for Learning} 
Working memory is involved in all deliberate cognitive operations \parencite{oberauerBenchmarksModelsShortterm2018} and is especially important in maintaining task-relevant information during cognitive tasks \parencite{baddeleyWorkingMemory1974, danemanIndividualDifferencesWorking1980}. It is well known that the capacity to store or manipulate information in WM is limited, but the exact capacity or the mechanistic functioning of WM that cause the limited capacity are still being discussed \parencite[see.,][]{oberauerWhatLimitsWorking2016,cowanMagicalMysteryFour2010}. Individuals differ in WM capacity, and these differences are often associated with differences in many high-level cognitive abilities \parencite{danemanIndividualDifferencesWorking1980, unsworthWorkingMemoryFluid2014,unsworthDivisionShorttermWorking2007,miyakeHowAreVisuospatial2001, cowanCapacityAttentionIts2005} and academic performance \parencite{gathercoleWorkingMemoryAssessments2003,gathercoleWorkingMemorySkills2004}. 
%It is generally agreed that WM is closely linked to attention \parencite{miyakeModelsWorkingMemory1999} with early models of memory seeing attention as a filtering operation limiting the amount of information entering or remaining in memory \parencite{baddeleyWorkingMemory1974, broadbentPerceptionCommunication1958}. %wieso Attention besprechen? vlt eher bei ADHDs
%Complexity of items that need to be stored is a further limiting factor, as additional resources are needed to resolve the complexity of those items \parencite{erikssonNeurocognitiveArchitectureWorking2015}.
The limitation of WM capacity is even more pronounced in childhood, while capacity has been shown to increase linearly from age 4 to 15 in cross-sectional studies \parencite{bradyCompressionVisualWorking2009, gathercoleWorkingMemorySkills2004}. Furthermore, the development of WM capacity is marked by a large degree of variability between individuals \parencite{gathercoleDevelopmentMemory1998}. This variability may be due in part to the many aspects of working memory that seem to develop during childhood and adolescents, as described by \textcite{cowanWorkingMemoryUnderpins2014}. First, it is suggested that the capacity to store items itself grows throughout childhood and adolescents \parencite{andrewsCognitiveComplexityMetric2002}. Second, it has been argued that increased knowledge contributes to the observed increase in working memory performance. For example, \textcite{chiKnowledgeStructuresMemory1978} showed that children with knowledge of the chess game remembered the chess configurations better than adults without knowledge of the chess game. Lastly, processing speed \parencite[e.g.,][]{kailProcessingSpeedMental1994} and the availability of strategies are discussed as contributing to improving working memory performance during childhood and adolescents \parencite{cowanWorkingMemoryUnderpins2014}. Although it is not yet clear which factor is primarily driving the observed differences in working memory performance, all should be considered in the context of learning and education \parencite{cowanWorkingMemoryUnderpins2014}. For example, complex learning material could be presented in a way that is accessible with limited working memory capacity, in a way that facilitates the use of strategies, or presented in the context of previous knowledge of the children. 
The large corpus of research has shown the crucial link between working memory and learning. For example, children with learning disabilities or low academic performance often have low scores on working memory tests \parencite{sabolPatternsSchoolReadiness2012, stclair-thompsonExecutiveFunctionsAchievements2006}. Additionally, successful participation in a classroom environment requires processing and storing of new information. Higher-level cognitive abilities involved in processing allow individuals to form plans, initiate actions toward those plans and persevere until their completion, are commonly described as executive functions (EF) or as central executive (referring to the working memory model of \textcite{baddeleyWorkingMemory1974}). Prominently postulated EF abilities are shifting (e.g., shifting between task, operations, or mental states), updating (e.g., monitoring and encoding relevant information, updating content held in WM by replacing irrelevant content) and inhibition (e.g., deliberately inhibiting automatic responses) \parencite{miyakeUnityDiversityExecutive2000}. These abilities have been shown to strengthen during childhood and adolescents \parencite{bestDevelopmentalPerspectiveExecutive2010}.
The prefrontal cortex (PFC) is a brain area that is often suggested to play a major role in networks associated with EF \parencite{diamondNormalDevelopmentPrefrontal2002, fiskeNeuralSubstratesEarly2019}. PFC is suggested to be one of the final brain regions to fully mature in humans and to have delayed maturation with neurodevelopmental disorders such as attention deficit hyperactivity disorder \parencite{kolkDevelopmentPrefrontalCortex2022}.
Poor performance in executive function tasks where the processing and storage of verbal material is tested has been linked to underachieving in English and mathematics \parencite{gathercoleWorkingMemoryDeficits2000}. EF furthermore promotes successful switching and evaluation of new strategies deployed during tasks \parencite{bullExecutiveFunctioningPredictor2001}, as well as staying on a task and keeping track of one's goal \parencite{kaneWorkingmemoryCapacityControl2003}, which is essential in learning settings \parencite{cowanWorkingMemoryUnderpins2014}. Individuals with psychological disorders characterized by deficits in EF and WM are often found to struggle in academic settings \parencite{lonerganMetaanalysisExecutiveFunctioning2019,martinussenMetaanalysisWorkingMemory2005,semrud-clikemanNeuropsychologicalAspectsEvaluating2005,willcuttValidityExecutiveFunction2005}, further highlighting the importance of these cognitive functions for learning and academic performance. Obtaining academic skills such as reading and mathematics is crucial for many outcomes later in life, including monetary income, physical and mental health, and longevity \parencite{calvinChildhoodIntelligenceRelation2017,kuncelFactFictionCognitive2010,wrulichFortyYearsChildhood2014}. Recently, cognitive abilities (e.g., WM and EF) and academic achievement have been considered to influence each other in a bidirectional way, that is, schooling facilitates the development of cognitive abilities, and increased cognitive abilities improve academic achievement\parencite{pengDevelopmentAcademicAchievement2020}. The above mentioned findings indicate the importance of identifying individuals struggling in academic settings and understanding their struggles with learning and memory formation.

\subsubsection{Studying Processes Underlying Memory Formation}
Studying the underlying processes involved in memory formation was done mainly with mathematical modeling of cognitive processes that are of theoretical importance \parencite{kahanaFoundationsHumanMemory2012, schallBuildingBridgeBrain2004}. It is difficult to measure these processes directly, although past studies have used measures such as response time to infer upon processes underlying memory formation \parencite[e.g.,][]{garnerOperationismConceptPerception1956}. With the ability to monitor neurophysiological processes in near instantaneous fashions, an opportunity has emerged to directly map neural and mental processes \parencite{schallBuildingBridgeBrain2004}.  Relying solely on behavioral data, such a direct link between mental and neural processes is difficult to make, as behavioral data reflect the final output of many cognitive processes \parencite{luuNeurophysiologicalMeasuresBrain2009} and can be confounded by numerous biases \parencite{webbUnobtrusiveMeasures1999}. 
Neurophysiological measures such as electroencephalogram (EEG) have been shown to be appropriate measures for assessing learning \parencite{tingaNoninvasiveNeurophysiologicalMeasures2019} and could provide novel insights about the underlying processes that contribute to task performance and possibly explain differences in performance\parencite[e.g.,][]{adamContributionAttentionalLapses2015,kimAdultADHDWorking2014}. Additionally, monitoring processes during learning in an online fashion can also be used to evaluate interventions with the aim of facilitating the learning process (e.g., napping vs. playing mobile games) \parencite{chiangEEGBasedDetectionModel2018}. Thus, incorporating such neurophysiological measurements could provide an opportunity for understanding why some individuals struggle with learning and identifying interventions that facilitate learning. 

\subsection{Electroencephalogram (EEG) and Event-related Potentials (ERP)}
EEG refers to electrical activity recorded from electrodes placed on the head surface. This electrical activity is generated by synaptic excitations of a large population of mainly pyramidal neurons with the same polarity and orientation in the cerebral cortex \parencite{luckIntroductionEventrelatedPotential2014,teplanFUNDAMENTALSEEGMEASUREMENT2002}. This summation of many postsynaptic potentials can be measured with a precision of milliseconds, which is a big advantage of EEG \parencite{lightElectroencephalographyEEGEventRelated2010}. 
There are multiple methods that allow one to extract useful information from the EEG signal. 
The computing of event-related potentials (ERP) is one of such methods and enables the extraction of brain activity associated with mental processes that occur in response to experimentally defined events \parencite{luckIntroductionEventrelatedPotential2014, surEventrelatedPotentialOverview2009}. Such events are, for example, the onset of a stimulus. ERPs are calculated from EEG recording by time locking the signal to an event of interest resulting in EEG epochs that can be averaged in a next step \parencite{luckIntroductionEventrelatedPotential2014}.  As described in \textcite{luckIntroductionEventrelatedPotential2014}, the assumption behind averaging these EEG epochs is that it results in random event-unrelated positive or negative fluctuations in voltage being canceled out. Therefore, the assumption is that brain activity that is consistently followed after the event remains after averaging, while random fluctuations that are not related to the event are canceled out. The resulting ERP waveform is described in terms of a sequence of positive and negative voltage peaks relative to a reference electrode. These peaks are characterized by amplitude and latency and have traditionally received the greatest interest in research \parencite{luckIntroductionEventrelatedPotential2014, luckOxfordHandbookEventrelated2012}. Early appearing peaks in the ERP waveform are often described as being exogenous or sensory, since they necessarily appear after an event and vary with physical parameters of an event (e.g., stimulus) \parencite{luckIntroductionEventrelatedPotential2014, surEventrelatedPotentialOverview2009}. Later appearing peaks in the waveform are seen as cognitive or endogenous, as they are related to high-order cognitive processes \parencite{luckIntroductionEventrelatedPotential2014, luckOxfordHandbookEventrelated2012, surEventrelatedPotentialOverview2009}. It should be noted that the distinction between exogenous and endogenous peaks is considered helpful but problematic because peaks can be influence by endo- and exogenous factors \parencite{luckIntroductionEventrelatedPotential2014}.
A highly studied peak is P300, first described by Sutton and colleagues after finding a larger positive deflection around 300 ms after presenting stimuli with unpredictable modality (i.e., visual or auditory) compared to stimuli with predictable modality \parencite{suttonEvokedPotentialCorrelatesStimulus1965}. 

\subsubsection{P300}
P300 or P3b refers to a positive deflection in amplitude of the ERP waveform that typically occurs 250-500 ms after the onset of an event \parencite{polichUpdatingP300Integrative2007}. Although the time at which the peak occurs (i.e., latency) depends on age, stimulus, modality, task conditions, and more \parencite{polichCognitiveBiologicalDeterminants1995}. This peak in relative amplitude was first described by \textcite{suttonEvokedPotentialCorrelatesStimulus1965} by demonstrating that stimuli with unpredictable modalities (i.e. visual or auditory) elicited larger positive amplitudes around 300 ms after presentation compared to stimuli with predictable modalities. The authors concluded that P300 reflects a correlate of stimulus uncertainty (or expectancy) and that this effect was independent of modality. Later,  \textcite{squiresTwoVarietiesLonglatency1975} found that presenting deviant auditory stimuli in a series of frequent auditory stimuli without instruction to react to the deviant stimuli elicited a P300-like peak with more of a frontal maximum, which was referred to as P3a to distinguish it from the more parietal located P3b (from here on referred to as P300). P3a is believed to reflect an automatic orientation response to interesting information and is sometimes referred to as novelty P3 \parencite{esceraInvoluntaryAttentionDistractibility2000, polichUpdatingP300Integrative2007}.
It is difficult to pinpoint the neural sources that generate the P300 signal because it is an inverse problem. An inverse problem refers to estimating unknown parameters given some observation or data where no unique solution is possible, rather there can be an infinite number of combinations of parameters that fit the observed data equality well \parencite{bailletForwardInverseProblems2013}.  In the context of EEG these unknown parameters are “anatomical location and current-flow orientation in space and amplitude variations in time” of neural sources generating the EEG signal \parencite[page. 1]{bailletForwardInverseProblems2013}. Still, research studying individuals with lesions has suggested that the temporal-parietal junction contributes to the P300 signal and with lesion in that area, P300 amplitudes are greatly reduces \parencite{knightContributionsTemporalparietalJunction1989, verlegerReductionP3bPatients1994}.

\subsubsection{Modulating Factors of P300}
Since its first observation, a lot of research has been conducted, specifying under which circumstances P300 would be elicited or altered, and exploring the functional significance behind this peak.  It has been suggested that P300 is an endogenous correlate of cognitive processing of stimulus information rather than reflecting brain activity elicited by the physical parameters of the stimulus presented \parencite{hopfingerInteractionsEndogenousExogenous2006,laceyCognitiveModulationTimedependent1980, pictonP300WaveHuman1992, pritchardPsychophysiologyP3001981}. It should be noted that physical changes in stimuli can elicit changes in P300 amplitude, but these changes depend on whether participants detect or have prior knowledge of these changes \parencite{donchinCognitivePsychophysiologyEndogenous1978}.
In subsequent research, oddball paradigms were often used to reliably elicit P300 amplitudes. Typical oddball paradigm tasks are where participants have to discriminate between a frequent stimulus and infrequent stimulus, the latter prompting a response from the participant (e.g. covert counting) \parencite{donchinCognitivePsychophysiologyEndogenous1978}. Additionally, variants of the oddball paradigm with single stimulus or three stimulus variants have been robustly shown to elicit P300 amplitudes \parencite{conroyNormativeVariationP3a2007, walshRelationshipP3bSingletrial2017}. A consistent finding in oddball paradigms has been the inverse relationship between the amplitude of P300 and the probability of which each target stimulus appears \parencite{donchinCognitivePsychophysiologyEndogenous1978}. \textcite{duncan-johnsonQuantifyingSurpriseVariation1977} found this effect across the entire probability range from 0.10 to 0.90, but crucially, this depends on the subjective probability assigned to the event by the participant \parencite{donchinCognitivePsychophysiologyEndogenous1978,pritchardPsychophysiologyP3001981}. Furthermore, only task-relevant stimuli elicited a large P300 amplitude \parencite{duncan-johnsonQuantifyingSurpriseVariation1977,polichUpdatingP300Integrative2007}. For example, \textcite{donchinAveragedEvokedPotentials1967} created an experimental setting in which a flash of light randomly appeared superimposed on a background of a circle randomly alternating with a square. One group was instructed to attend to the background, while the other group was instructed to attend to the flash of light. Only the attend condition (i.e., background or flash of light) elicited a P300 amplitude, suggesting the importance of attention and task relevance for the P300 amplitude \parencite{donchinCognitivePsychophysiologyEndogenous1978,pritchardPsychophysiologyP3001981}. 
Early on, \textcite{donchinSurpriseSurprise1981} mentioned that the characteristics of P300 are very similar to those of orienting reflex (OR), specifically, task relevant and improbable events (or stimuli) tend to elicit OR. OR refers to an immediate behavioral or neurophysiological response to changes in the environment and was first postulated by \textcite{pavlovConditionedReflexesInvestigation1927}. As described in \textcite{rushbySeparationComponentsLate2005}, \textcite{sokolovPerceptionConditionedReflex1963} later proposed that with repeated presentation of a stimulus, an according cortical representation (neuronal model) is formed and new stimuli that do not match this representation elicit an OR. Relating P300 and OR, \textcite{donchinSurpriseSurprise1981} suggested that the process of updating one’s neuronal model is manifested in the amplitude of P300, furthermore, directly linking it to memory and learning processes through surprise and orienting reflex. Subsequent studies have aimed to test the context updating theory and its prediction that “events that elicit a P300 are remembered better than events that do not elicit a P300” \textcite[page. 509]{donchinSurpriseSurprise1981}.

\subsubsection{P300 and Memory}
In a seminal study, \textcite{karisP300MemoryIndividual1984} investigated whether P300 during encoding is related to subsequent recall performance as predicted by the context updating theory. The authors found that for participants using simple learning strategies (i.e., mnemonic), the amplitude of P300 predicted later recall performance, but for participants using more complex strategies to organize the learning material, P300 did not predict later recall but rather a frontal positive slow wave did \parencite{karisP300MemoryIndividual1984}. This effect of rehearsal strategy on the predictability of subsequent recall of P300 was further confirmed in a within-subject design manipulating the instructed rehearsal strategies, leading the authors to suggest future studies to consider paradigms “that narrow the range of variables that can affect recall to variables that are also likely to affect the amplitude of the P300” so that the relationship of P300 and memory can be better understood \parencite[page. 300]{fabianiEffectsMnemonicStrategy1990}.
\textcite{johnsonjr.P300LongTermMemory1985} used a study-test paradigm to test the association of P300 with long-term memory acquisition. In this study, participants learned a list of words that was repeated four times with a test phase after each repetition. The test phase consisted of a recognition test in which learned words were mixed with an equal number of random new words (each test phase included different new words). The authors found that the amplitude of P300 increased during the study and test phases with an accompanying increase in recognition performance and that later recognized words elicited larger P300 compared to unrecognized words, although this difference did not reach significance \parencite{johnsonjr.P300LongTermMemory1985}. The difficulty of the task can directly influence P300 amplitude \parencite{courchesneChangesP3Waves1978, kokUtilityP3Amplitude2001} and was mentioned by \textcite{johnsonjr.P300LongTermMemory1985} as a possible reason for not finding a significant difference between P300 and later recognized words, as the list of words was rather long. However, in many studies, the use of P300 has been found to predict the subsequent memory performance of words \parencite[see.,][]{pallerERPsPredictiveSubsequent1988}. 
Using a study-test paradigm with repeated presentation, as was done in \textcite{johnsonjr.P300LongTermMemory1985}, could provide valuable information when connecting P300 with memory processes, as repeating the same sequence strengthens memory traces, which is reflected in improved recall performance \parencite{hintzmanRepetitionMemoryEvidence1971}. This is further mentioned by \textcite{ruggEventrelatedPotentialsStimulus1994} who suggested that changes in neural response to repeated presentation of stimulus indicate the formation of an internal representation of that stimulus. In light of the context updating theory \parencite{donchinSurpriseSurprise1981}, one would expect P300 to decrease with repetition, since learned material becomes increasingly expected and the internal representation formed of the sequence needs less updating. The findings of \textcite{johnsonjr.P300LongTermMemory1985} that the amplitude of P300 increased with repetition are incongruent with this assumption, although other studies using repeated presentation have shown that P300 decreases as events become expected \parencite{courchesneChangesP3Waves1978} and is also a consistent finding from oddball paradigm. Furthermore, the P300 amplitude is considered to be influenced by the sequence of immediately preceding stimuli \parencite{squiresEffectStimulusSequence1976} and since the sequence in the recognition phase differed from the sequence in the study phase, this may have influenced P300 amplitudes.  Using a list of words to study memory could lead to complexities that systematically vary with age, for example, the use of strategies during encoding \parencite{schneiderDevelopmentYoungChildren2004}.  Since P300 amplitude has been linked to memory encoding that promotes successful memory rather than directly to memory itself \parencite{fernandezRealtimeTrackingMemory1999, polichUpdatingP300Integrative2007}, using a list of words to learn as in \textcite{johnsonjr.P300LongTermMemory1985} may provide more variability in encoding strategies like chunking \parencite{nogueiraLatePositiveSlow2015} or imagery \parencite{zahediImpactImageryStrategy2012} between individual items as well as between participants with varying age. Incorporating paradigms with highly simplified stimuli could minimize variables like encoding strategies that possibly vary between items and are known to influence P300 amplitude \parencite[e.g.,][]{karisP300MemoryIndividual1984}. \textcite{steinemannTrackingNeuralCorrelates2016} used a visual sequence learning paradigm (see Chapter\ref{Methods}) in which participants learned a sequence of highly simplified stimuli with minimal semantic value (i.e. flashing circles) through repetition with regular behavioral recall. With stimuli only differing in their spatial locations, this paradigm allowed differentiating between endogenous variability in encoding or memory formation and interstimulus variability within and between subjects \parencite{steinemannTrackingNeuralCorrelates2016}. Furthermore, \textcite{steinemannTrackingNeuralCorrelates2016} mentioned an additional strength of this paradigm is to track the memory traces within individual stimuli as they gradually transit from unknown over newly learned to fully committed to memory. The authors hypothesized that stimuli fully committed to memory become expected during subsequent presentation, which in turn would manifest itself in lower surprise-regulated P300 amplitudes \parencite{steinemannTrackingNeuralCorrelates2016}. This hypothesis could be confirmed as well as predicting successful learning with P300 amplitude, which led the authors to discuss the potential of their findings as a diagnostic tool for learning disabilities, although more studies are needed in such target populations \parencite{steinemannTrackingNeuralCorrelates2016}. 

\subsection{Clinical Application}
ERP components provide useful information in clinical settings as near real-time measures of covert information processing, such as encoding processes \parencite{friedmanEventrelatedPotentialERP2000}. For example, \textcite{rigginsP300DevelopmentInfancy2020} noted that with people whose expressive language or motor skills are impaired, the ability to rely on ERP components would be valuable. Additionally, it was suggested that P300 could in the long run provide information on the identification of individuals from vulnerable groups at risk of developing childhood disorders \parencite{duncanEventrelatedPotentialsClinical2009}. A study exploring a similar objective was conducted by \textcite{willnerRelevanceNeurophysiologicalMarker2015}. Within a socioeconomically disadvantaged population, the authors found an association between higher P300 amplitudes and adaptive learning-related behaviors (e.g. sit still, listen without being distracted, less frustration) in kindergarten. These behaviors, in turn, were related to an increase in academic performance from kindergarten to 1st grade \parencite{willnerRelevanceNeurophysiologicalMarker2015}. Identifying individuals at risk could facilitate early interventions, as experiencing academic underachievement can increase the risk of developing externalizing behaviors and substance dependence \parencite{crumAssociationEducationalAchievement1998, hinshawExternalizingBehaviorProblems1992}.
Attention-Deficit/Hyperactivity Disorder (ADHD) is a disorder characterized by impaired attention, hyperactivity, or impulsivity \parencite{swansonAttentiondeficitHyperactivityDisorder1998} and is one of the most common disorders in children and adolescents with a prevalence estimated at 5.3\% \parencite{polanczykEpidemiologyAttentiondeficitHyperactivity2007}. ADHD has often been shown to have an impact on academic success \parencite{frazierADHDAchievementMetaAnalysis2007}, with deficits in working memory (WM) being one possible reason \parencite{kimAdultADHDWorking2014, martinussenMetaanalysisWorkingMemory2005}. Meta-analyses investigating P300 differences in ADHD revealed lower amplitudes and longer latencies compared to a control group and further discuss P300 as a potential biomarker of ADHD, although substantial heterogeneity in effect sizes and other factors limit this conclusion \parencite{kaiserEarlierLaterCognitive2020,szuromiP300DeficitsAdults2011}. These meta-analyses only examined P300 in GoNoGo paradigms, although it would be valuable to assess P300 in learning and memory tasks. Especially, since ADHD has a high cooccurrence with specific learning disorders (SLD) \parencite{boadaUnderstandingComorbidityDyslexia2012}. SLD is used to describe individuals with significant difficulties in learning one or more academic domains (e.g., reading, writing, mathematics) \parencite{phamSpecificLearningDisorders2015}. The diagnosis of SLD has been described as controversial, and a traditional criterion is a discrepancy between performance on standardized achievement tests and IQ. This criterion has been criticized for its low validity and reliability as diagnostic criteria \parencite{fletcherEvidencebasedAssessmentLearning2005}. Response-to-intervention (RTI) is an alternative approach to identify individuals at risk marked by inadequate response to instructions or interventions \parencite{vaughnRedefiningLearningDisabilities2003}. This approach promotes a move from a deficit model to a risk model, as in early detection of individuals at risk rather than waiting for academic deficits to occur, and also focuses on ongoing progress-monitoring as well as detection in early stages like kindergarten \parencite{vaughnRedefiningLearningDisabilities2003}. P300 could play a potential role in such identification and guide interventions as found in \textcite{wieckiModelBasedCognitiveNeuroscience2015}. Still, since \textcite{wieckiModelBasedCognitiveNeuroscience2015} used a paradigm which did not allow for a direct connection between P300 and learning rather this link was mediated through learning-related behavior. It may be important to use paradigms in which individuals are instructed to learn using repeated presentations to create a direct link between P300 and learning or memory formation.  As mentioned, such a paradigm was used by \textcite{steinemannTrackingNeuralCorrelates2016} which allowed P300 to directly be connected to successful learning. \textcite{strzelczykNeurophysiologicalMarkersSuccessful2022} extended these findings in a population of healthy older adults. It is still unclear whether this relationship holds in a population of children and adolescents with psychiatric and learning disorders. 
As \textcite{vaughnRedefiningLearningDisabilities2003} mentions, more research is needed to provide validated methods for testing responsiveness to instructions. P300 amplitudes are associated with encoding processes of successful learning \parencite{polichUpdatingP300Integrative2007} and have been shown to help improve attention and behavioral performance with a P300 based neurofeedback training \parencite{arvanehP300BasedBrainComputerInterface2019}. Therefore, in the future P300 could potentially be helpful as tools for monitoring cognitive processes and responsiveness to intervention or instructions even in  communication-impaired individuals \parencite{connollyApplicationCognitiveEventrelated2000}. Multiple studies have suggested the utility of P300 in academic settings to inform early intervention or disentangle individual differences in cognitive performance from impacting factors such as test anxiety \parencite{priviteraUtilityP3Neuromarker2022, vonderembseTestAnxietyEffects2018,willnerRelevanceNeurophysiologicalMarker2015}. Still, none of the above mentioned studies have used paradigms where explicit learning over multiple repetitions was required \parencite[see.,][]{priviteraUtilityP3Neuromarker2022}. 
More research with paradigms assessing learning is needed while considering variables that have been shown to influence P300 amplitude  \parencite{cuiP300AmplitudeLatency2017,duncanEventrelatedPotentialsClinical2009,polichCognitiveBiologicalDeterminants1995,rigginsP300DevelopmentInfancy2020,surEventrelatedPotentialOverview2009,vandinterenP300DevelopmentLifespan2014}. For example,  P300 has often been found to be altered in psychiatric disorders \parencite{polichClinicalApplicationP3002004,surEventrelatedPotentialOverview2009} but has been described as unspecific and not relating to a single diagnosis \parencite{duncanEventrelatedPotentialsClinical2009}. Moreover, P300 amplitude have been shown to have high heterogeneity within a diagnosis \parencite[e.g.,][]{cuiP300AmplitudeLatency2017,kaiserEarlierLaterCognitive2020, martinussenMetaanalysisWorkingMemory2005}. This is not surprising since heterogeneity within diagnosis is a known phenomenon \parencite{astleAnnualResearchReview2022,kaiserEarlierLaterCognitive2020}.

\subsection{Research Gap}
As previously mentioned, P300 has the potential to inform about the cognitive processes that underlie successful learning without relying on behavioral reports and in near real-time fashion, which could be helpful in studying children and adolescents who struggle with learning \parencite{rigginsP300DevelopmentInfancy2020,steinemannTrackingNeuralCorrelates2016,strzelczykNeurophysiologicalMarkersSuccessful2022,tingaNoninvasiveNeurophysiologicalMeasures2019}. In healthy young and old adults, a direct link between P300 amplitudes and successful learning has already been established \parencite[e.g.,][]{steinemannTrackingNeuralCorrelates2016,strzelczykNeurophysiologicalMarkersSuccessful2022}. Visual sequence learning paradigm (VSLP) in which participants learn a sequence of highly simplified stimuli (i.e., flashing circles) with little semantic content has been a main contributing factor in establishing this link. However, it is unclear whether the same link exists in a population with psychiatric and learning disorders.
Therefore, this thesis aims to close this gap by investigating whether P300 as a neural correlate of successful learning can be found in individuals with psychiatric and learning disorders using the visual learning paradigm in combination with event-related potentials. This thesis is closely aligned with a study by \textcite{strzelczykNeurophysiologicalMarkersSuccessful2022} that extended the findings of \textcite{steinemannTrackingNeuralCorrelates2016} to a sample of healthy older adults. 
With P300 amplitude being sensible to stimulus-bound subjective degree of surprise (or expectation) bound to stimulus \parencite{donchinSurpriseSurprise1981,duncan-johnsonQuantifyingSurpriseVariation1977,marsTrialbyTrialFluctuationsEventRelated2008,suttonEvokedPotentialCorrelatesStimulus1965}, this component provides an opportunity of linking P300 to gradually increasing knowledge of the sequence in VSLP over the course of successful learning \parencite{steinemannTrackingNeuralCorrelates2016}. Given that participants were instructed to learn a sequence of spatial locations (see \nameref{VSLP}) it was expected that the knowledge of the presented sequence (i.e., Knowledge Index: see equation \ref{eq:KI}) increases over the course of learning: 
\begin{enumerate}
     \item[H1)] Behavioral performance increases over the course of learning.
\end{enumerate}


During childhood and adolescents, cognitive abilities increase with age; therefore, it was predicted that behavioral performance increases with increasing age during childhood and adolescents: 

\begin{itemize}
    \item[H2)] Behavioral performance increases with age during childhood and adolescents. 
\end{itemize}


On a neurophysiological level, the amplitude of P300 was expected to decrease monotonically over the course of learning, from stimuli being unknown, to newly learned and fully committed to memory (i.e., known). Similar patterns in P300 decrease were expected in all age ranges, but amplitudes and latency differ as a function of age. Hence, the fourth hypothesis was that in children and adolescents P300 amplitudes latency decreases with age:

\begin{itemize}
    \item[H3)] P300 amplitude decreases monotonically over the course of learning;
    \item[H4)] P300 amplitude decreases with increasing age during childhood and adolescents.
\end{itemize}


Finally, it was investigated whether neurophysiological measures could predict successful learning within individuals. Therefore, it was hypothesized that increased behavioral performance is associated with decreased P300 amplitudes: 
\begin{itemize}
    \item[H5)] Increase in behavioral performance is associated with a decrease of P300 amplitude. 
\end{itemize}

In a next step, it was explored whether P300 amplitude decreases over the course of learning are associated with transdiagnostic constructs. This investigation was exploratory in nature, without concrete hypotheses. 
