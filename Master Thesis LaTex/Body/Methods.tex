\section{Methods} \label{Methods}

\subsection{Data Acquisition}
Data was acquired via the 1000 Functional Connectomes Project and its International Neuroimaging Data-sharing Initiative (FCP/INDI) (see \nameref{DataCitation}) \parencite{alexanderOpenResourceTransdiagnostic2017}. Data processing and analysis was performed with MATLAB R2021b (The MathWorks, Inc., Natick, Massachussets, United States) and RStudio 4.2.2 (R Core Team). 

\subsection{Participants}
The sample used in this thesis was obtained from the Child Mind Institute Healthy Brain Network (HBN) Biobank. HBN recruits with a community-referred strategy in New York City with the primary goal of generating a data set that represents the wide range of heterogeneity and impairment that occurs in developmental psychopathology. Upon completion of the study, participants receive comprehensive diagnostic evaluation reports, referral information, up to three in-person feedback sessions, and modest financial compensation \parencite{alexanderOpenResourceTransdiagnostic2017}. 
Of the 2'091 participants for whom relevant data used in this thesis were available, 497 were excluded (see figure \ref{fig:Diagram}). Thus, the final sample consisted of 1'594 participants with an age range of 5.04 to 21.90 years of age. The mean age was 10.77 with a standard deviation of 3.44 (see table \ref{tab:demo} for demographics). Many of the participants received or already had at least one diagnosis (n = 1'437) while 157 participants had no diagnosis (see table \ref{tab:diag}). The most frequent diagnosis can be found in table \ref{tab:diag}.

\begin{figure}[H] 
\centering
\begin{tikzpicture}%
  [data/.style=
    {draw,minimum height=0.7cm,minimum width=2cm,align=center},
   filter/.style=
    {draw,minimum height=1.3cm,minimum width=3cm,align=center},
   database/.style=
    {draw,minimum height=1.5cm,minimum width=3cm,align=center},
   flow/.style={thick,-stealth},
   apply/.style={}
  ]
  \node[database] (db) {Healthy Brain Network Biobank (HBN)\\$N = 2'999$};
  \node[data,below=of db] (d1) {Full VSLP Data \\ $N=2'091$};
  \node[data,below=of d1] (d2) {Total Sample Size:\\$N=1'594$\\$Trials = 58'060$};
  \draw[flow] (db) -- (d1);
  \draw[flow] (d1) -- coordinate(d1d2) (d2);
  \node[filter,right=of db] (excl) {Exclusions};
  \node[filter] (f1) at (d1d2-|excl) {Bad EEG Rating ($N=224$)\\
                                    <50\% artifact free trials ($N=177$)\\
                                    No demographics ($N = 91$)\\
                                    >10 years old with seq. length 7 ($N=5$)};
  \draw[apply] (d1d2) -- (f1);

\end{tikzpicture}
\caption{Sample Size and Exclusion Criteria.} \label{fig:Diagram}
\end{figure}


\subsection{Visual Sequence Learning Paradigm} \label{VSLP}
The visual sequence learning paradigm (VSLP) \parencite{moiselloNeuralActivationsVisual2013} was one of many tasks applied in the HBN study \parencite[see][]{alexanderOpenResourceTransdiagnostic2017}. During VSLP neurophysiological measures were simultaneously recored, which together with behavioral task performance provided the main data for this thesis.   
In the VSLP participants had to learn a fixed sequence of visual stimulus positions presented one by one. The possible locations were constantly marked by circular outlines with the target stimuli (i.e., to be learned locations) being filled white circles. These possible locations lay equidistant around a ring of fixed eccentricity. The target stimuli appeared for 200 ms each with 1300 ms between them (i.e. inter-stimulus interval). During the presentation, participants were asked to fixate on a central fixation point. To account for floor effects, the length of the sequence was shorter, and the number of possible locations was less for most participants under nine years. Specifically, this meant a sequence of 8 target stimuli with 6 possible locations. For older participants, the sequence consisted of 10 target stimuli and 8 possible locations.  In total, the same sequence was presented five times. After each sequence presentation, participants were asked to recall the locations in a test phase by clicking on them with a computer mouse. The test phase had no time restriction and no feedback was given, whether the correct locations were recalled \parencite{langerResourceAssessingInformation2017}.
Multiple studies with an interest in memory formation and learning processes have applied VSLP \parencite[e.g.,][]{steinemannTrackingNeuralCorrelates2016,strzelczykNeurophysiologicalMarkersSuccessful2022}. Repeated presentation of the same sequence enables the ability to observe learning progress during the recall or test phase. Since VSLP uses highly simplified stimuli, the factors that could influence learning success are minimized \parencite{steinemannTrackingNeuralCorrelates2016}. 

\subsection{Behavioral Data}
Individual target stimuli during each sequence presentation were categorized based on the response of the participants in the test phase. 
Newly learned category (NL) was assigned when the location of the target stimulus was correctly recalled for the first time. Known (K) categorized target stimuli that were correctly recalled at least two consecutive sequence repetitions. Unknown category (UN) classified target stimuli that were not recalled correctly. Finally, the forgotten category (F) was assigned to target stimuli that were NL or K in previous sequence repetitions, and an incorrect recall was made in the current repetition. If a forgotten target stimulus was correctly recalled in a subsequent presentation, the category would change from F to NL. If in a subsequent presentation, a previous forgotten target stimulus is recalled incorrectly again, it becomes unknown. On average, the forgotten category is assigned 9.7\% per participant (see table \ref{tab:avgTLS}).
Furthermore, for every sequence presentation, a knowledge index (KI) was calculated to track the progress of cumulative knowledge of the sequence. The KI was calculated for each sequence presentation of a given participant and was defined as: 
\begin{equation}\label{eq:KI}
    \textrm{Knowledge Index}_{(P,sr)} = \frac{nK_{(P,sr)}+nNL_{(P,sr)}}{nT_{(P,sr)}}
\end{equation}


where P is participant, sr is sequence repetition, K is known category, NL is newly learned and T is the total artifact free trials (see \ref{EEGdatapre}). The KI differed from the one used in \textcite{steinemannTrackingNeuralCorrelates2016} since the paradigm was slightly changed (see \nameref{SuppM}).
Another behavior performance measure used was the learning index (LI). LI is the proportion of newly learned stimuli in a sequence repetition for a given participant and was defined as a ratio of NL stimuli to the total number of stimuli (again only considering artifact free trials): 
\begin{equation} \label{eq:LI}
    \textrm{Learning Index}_{(P,sr)} = \frac{nNL_{(P,sr)}}{nT_{(P,sr)}}
\end{equation}


\subsection{EEG Data Acquisition}
The HBN study, described in \textcite{alexanderOpenResourceTransdiagnostic2017}, used a 128-channel EEG geodesic hydrocel system by Electrical Geodesic Inc. (EGI) with a sampling rate of 500 Hz to record the EEG data. Cz (vertex of the head) was the recording reference and the impedance of each electrode was kept below 40 kOhm. 

\subsection{EEG Data Preprocessing} \label{EEGdatapre}
Since the EEG data was provided in raw format by HBN, the data had to be prepared for further analysis. Data preprocessing was performed using Automagic 2.6, a MATLAB based toolbox for reliable and objective preprocessing in an automatic way \parencite{pedroniAutomagicStandardizedPreprocessing2019}.
Automagic used the PREP pipeline for large-scale EEG analysis \parencite{bigdely-shamloPREPPipelineStandardized2015} to detect noisy or outlier channels. Four criteria are used for detection: extreme amplitudes (deviation exceeding a z-score of 5.00), low correlation with any other channel (correlational threshold: 0.4), lack of predictability by other channels, and unusual high frequency noise (deviation of signal-to-noise ratio exceeding a z-score of 5.00). More detailed descriptions of these processes can be found in \textcite{pedroniAutomagicStandardizedPreprocessing2019} and \textcite{bigdely-shamloPREPPipelineStandardized2015}. Next, automagic handled sweating and power line artifacts by performing a high-pass filter (0.05 Hz) using $pop\_eegfiltnew$ from EEGLAB \parencite{widmannFilterEffectsFilter2012} and a zapline filter (50.00 Hz) using $nt\_zapline$ \parencite[NoiseTools,]{decheveigneZapLineSimpleEffective2020}. Further artifacts caused by, for example, muscle movement were identified and removed using Independent component Label (ICLabel) \parencite{pion-tonachiniICLabelAutomatedElectroencephalographic2019}. 


Following automagic preprocessing procedures, a 45 Hz low-pass filter was applied using $pop\_eegfiltnew$ because only ERP signals are of interest and are mainly below 30 Hz \parencite{luckIntroductionEventrelatedPotential2014}. Next, EEG data was re-referenced to the average of all electrode sites with the $pop\_reref$ function of EEGLAB. Using the average as a reference instead of a single electrode site minimizes artifacts linked to that specific electrode site \parencite{dienIssuesApplicationAverage1998}.
In a second step, the EEG signal was segmented into trials using triggers that marked the appearance of target stimuli (i.e., stimulus onset). The duration of segments was 900 ms (100 pre stimulus onset to 800 ms post stimulus onset). 
At this point two data sets were created, one for visualization purpose and a second data set used to extract P300 amplitudes for statistical analysis. For the first data set, the segmented EEG signals were corrected for baseline using the $pop\_rmbase$ function and a time window of 100 ms pre stimulus onset. The segments of each electrode were compared to an amplitude threshold of -+ 90 microvolts to mark further artifacts. If at any point during a segment this threshold was exceeded, then the trial is marked as artifact and was removed from both data sets. Second, participants who had more than 50\% of their total trials marked as artifacts were excluded from both data sets (see figure \ref{fig:Diagram}). The cutoff point of 50\% was chosen because having a higher cutoff point, for example 75\% would have led to over 1'000 subjects being excluded. 
For the second data set (used for P300 amplitude extraction), the EEG segments of a cluster of centro-parietal electrodes were averaged together, while discarding EEG segments from other electrodes. This cluster consisted of the following electrodes: E54, E55, E61, E62, E78, E79 located at the centro-parietal area. 
Additionally, in this data set no baseline correction was applied. Rather, baseline was included in linear mixed effects models, following the approach of allowing the amount of baseline correction to be determined by statistical model as suggested by \textcite{aldayHowMuchBaseline2019}.


\subsection{Electrode Sites Selection}
The same electrode cluster was selected as in a previous study that used the same paradigm in young and old adults \parencite{strzelczykNeurophysiologicalMarkersSuccessful2022}. In a subsequent step, the grand average scalp topographies of all trials between 150 and 550 ms with a 50 ms step were plotted for different age groups (see figure \ref{fig:elecsel} in \nameref{SuppM}). By visual inspection of these plots, the electrode cluster was considered appropriate in capturing the P300 amplitude for all age groups. Averaging the signal from an electrode cluster where ERP effects are expected has been discussed as a temporary solution to possible differences in the scalp distribution of ERP in young children \parencite{brookerConductingEventRelatedPotential2020}.

\subsection{P300 Extraction}
To extract the P300 amplitude, participants were grouped by age into six bins of two years with the exception of participants 17 years and older, which formed a seventh group with a bin size of 5 years. Reasoning behind the seventh group was to decrease the signal-to-noise ratio, as there were fewer participants in this age range. A time window used to calculate the P300 amplitude was defined by averaging all trials in each age group, resulting in seven grand average ERP waveforms. Using all trials to select the time window (i.e., collapsed localizers) minimizes biases, since differences in conditions are not driving the selection of the time window \parencite{luckIntroductionEventrelatedPotential2014,luckHowGetStatistically2017}. The P300 peak was defined as the most positive amplitude in the time range typical for P300 (i.e., 300 - 500 ms post stimuli onset). Using the latency of the defined peak, a time window was calculated by adding and subtracting 50 ms. In a next step, these time windows were used to calculate the P300 amplitude in each single trial by averaging the voltage in the age appropriate time window (i.e., mean amplitude). Using mean amplitude helps against peak distortion or bias caused by signal noise \parencite{luckIntroductionEventrelatedPotential2014}.

\subsection{Conceptual Replication}
The first part of this thesis aimed at conceptually replicating findings from \textcite{steinemannTrackingNeuralCorrelates2016} and \textcite{strzelczykNeurophysiologicalMarkersSuccessful2022} . A conceptual replication is described as an attempt to test a fundamental idea or hypothesis behind the original study, but can vary in aspects such as experimental operationalization and design, independent and dependent variables and sample population \parencite{crandallScientificSuperiorityConceptual2016}. One main difference from the studies mentioned above was the different sample population. Whereas \textcite{steinemannTrackingNeuralCorrelates2016} investigated neuronal correlates of learning process and memory formation in a sample of healthy young adults and \textcite{strzelczykNeurophysiologicalMarkersSuccessful2022} compared a sample of healthy young adults with healthy old adults, this thesis focused on a sample of children and adolescents with various psychiatric disorders.

\subsubsection{Statistical Analysis}
The main analyses were performed with linear Mixed-Effects Models (LME). LME is a regression-based statistical method that is useful in analyzing hierarchically grouped data \parencite{bickelMultilevelAnalysisApplied2007}.
Event-related potentials are naturally grouped within individuals; therefore, it has often been advocated to apply LME in the analysis of ERP data \parencite{volpert-esmondUsingMultilevelModels2021,vossenMorePotentialStatistical2011}. 
To estimate the models, lme4 package \parencite{batesFittingLinearMixedEffects2015} in R was used with LmerTest \parencite{kuznetsovaLmerTestPackageTests2017} providing p values and the step function used for model selection. The step function follows a data-driven approach of iteratively deleting nonsignificant effects from a specified model. A step-down approach was used, in which the starting point is a full model that is compared to a reduced model using a likelihood-ratio test in an iterative process until the model with the best fit is found. This algorithmic process first determines the structure of random effects and proceeds with the structure of fixed effects \parencite{kuznetsovaLmerTestPackageTests2017}. Specifying the model as input for the step function was done by keeping the fixed effects structure as general as possible (i.e., allowing most fixed effects to interact with each other) and keeping the random part of the model parsimonious since having many random effects led to converging problems during model selection. The model formulas are expressed using the Wilkinson notation \parencite{wilkinsonSymbolicDescriptionFactorial1973}.

\subsubsection{Behavioral Analysis} \label{sec:Behavrioalana}
First, learning performance was analyzed descriptively by plotting behavioral measures (i.e., KI: \ref{eq:KI} and LI: \ref{eq:LI}) over repetitions. More specifically, the group averages of KI and LI with their standard errors were plotted across repetitions to display learning effects and potential age effects. Next, two LME (\ref{eq:mKI}) were specified with KI and LI as dependent variables (continuous: 0-1), repetition (ordinal: 0-4), age (continuous: grand mean centered) and sequence length (categorical: 7 and 10, 10 as reference category) as independent variables. The random effects structure consisted of a random intercept for every participant and a random slope for repetition including their correlation. To help with interpretation of interaction effects as well as having a meaningful intercept, the variables repetition and age were linearly transformed: repetition was subtracted by one, age was mean centered. Therefore, the intercept represented the first repetition and age equal to zero represents an average-aged person. This becomes important for the interpretation of the main effects of explanatory variables that have a significant interaction with age or repetition as these main effects should be interpreted together with the interaction as a system \parencite{hoxMultilevelAnalysisTechniques2017}. Overall, linear transformation does not change the model fit and residuals, and so models with or without linear transformation are equivalent only differing in what part of the model is emphasized \parencite{hoxMultilevelAnalysisTechniques2017}. On the other hand, the random variance of the intercept can be influenced by transformation when a random slope is included in the model \parencite{hoxMultilevelAnalysisTechniques2017}. 

\begin{equation}\label{eq:mKI}
KI \textrm{ or } LI   \sim  Repetition*Age*SequenceLength + (1+Repetition | Participant)
\end{equation}

\subsubsection{Neurophysiological Analysis}
\paragraph{P300 Amplitude Over Sequence Repetitions}
In the context of VSLP, the assumption was that learning a sequence increases its expectancy. As it is suggested that the P300 amplitude decreases as a function of expectancy, changes of P300 amplitude over repetition were analyzed. More specifically, it was hypothesized that over the course of leraning, P300 amplitude decreases. This hypothesis was approached through plots and LME. In the first step, the average P300 amplitude per sequence repetition was calculated for each participant (i.e., mP300). Second, to visualize possible age effects, mP300 was averaged within each age group and plotted across repetitions with error bars reflecting standard errors. Next, a linear mixed effects model was specified (\ref{eq:mP300}) as input for the step function. The specification was as follows: mP300 as dependent variable (continuous: 0-1), repetition (ordinal: 0-4), gender (categorical: male and female, with male as reference category), age (continuous: grand mean centered), sequence length (categorical: 7 and 10, 10 as reference category) and mBase (continuous) as independent variables. mBase was computed by averaging the baseline values of the trials in a given repetition. For every participant, a random intercept was included. 
\begin{equation}\label{eq:mP300}
mP300 \sim Repetition*Age*Gender*SequenceLength + mBase + (1| Participant)
\end{equation}

\paragraph{Predicting Successful Learning}
To further test the possible connection between P300 amplitude and knowledge index, a LME was specified (\ref{eq:mPre}) to investigate whether the differences in KI are explained through differences in mean P300 amplitude. The model was specified with KI as dependent variable (continuous: 0-1), mP300 (continuous), age (continuous: grand mean centered), gender (categorical: male and female, with male as reference category) and mBase (continuous) as independent variables. The model included an intercept as a random effect.

\begin{equation}\label{eq:mPre}
KI \sim mP300*Age*Gender +  mBase + (1 | Participant)
\end{equation}

\paragraph{P300 Amplitude Across Learning Categories}
In a final step, differences in P300 amplitude across learning categories (i.e., Unknown, Newly Learned, Known, Forgotten) were analyzed to directly tie the amplitude of P300 to the behavioral response at a single trial level. Since learning categories differentiate between first accurate recall of a spatial location (i.e., Newly Learned) and subsequent accurate recall (i.e., Known), these categories provide additional information which can be used to refer to how well a spatial location is learned or to what degree it is expected. Connecting P300 amplitude to learning categories on a single trial level is important, as only analyzing the P300 amplitude as an average of a sequence repetition can be influenced by habituation \parencite{ravdenHabituationP300Visual1998}. 
Topographical plots were created for the learning categories Unknown, Newly Learned and Known for four randomly selected age groups. For visualization, baseline corrected data set was used and all trials of each learning category were averaged together. Potential age effects were checked by plotting the average P300 amplitude and ERP's of learning categories across age groups the same age groups. In a next step, a LME model was specified (\ref{eq:mCat}). The model consisted of P300 as dependent variable, learning categories (categorical: UN,NL,K,F with NL as reference category), age, gender and base as independent variables, and a random intercept for each participant.  
\begin{equation}\label{eq:mCat}
P300 \sim LearningCategories*Age*Gender + base + (1 | Participant)
\end{equation}
Since LME only compares each learning category to the reference category, contrasts between the learning categories were computed using the emmeans package in R Studio while correcting for multiple comparisons using Bonferroni correction \parencite{lenthPackageEmmeans2019}. 

\subsection{Exploratory Analysis}
The second part of the thesis explored whether successful learning in VSLP and the degree of decrease in the amplitude of P300 over repetitions are associated with transdiagnostic constructs related to learning. 

\subsubsection{Knowledge Index and P300 Decrease}
For every participant, an average of their repetition based knowledge index was calculated.  
Second, a measurement reflecting the degree of a participants increase in expectancy of sequence was created. For this, a linear regression with P300 amplitude as dependent variable and repetition as independent variable was specified for each participant. The resulting slope coefficient of repetition was used in the exploratory part. 

\subsubsection{Questionnaires and Tests}
Since this part is exploratory in nature, no theoretical framework guided the inclusion of questionnaires. Instead, questionnaires that were filled out by most participants were considered and subscales of these questionnaires that have been found to be relevant for either learning, working memory test performance, or P300 amplitude were included. If available, standardized scores were used; otherwise, raw scores were included.

\subsubsection{SWAN}
The Strengths and Weaknesses of ADHD symptoms and Normal behavior scale \parencite[SWAN][]{swansonGenesAttentiondeficitHyperactivity2001}) provided scores for attention problems. SWAN has a bidirectional questionnaire design assessing problems but also strength of symptoms relevant in ADHD with a 7-point scale with endpoints: far below average and far above average \parencite{alexanderMeasuringStrengthsWeaknesses2020}. This questionnaire has been applied in multiple areas such as genetic studies, pharmacotherapeutics, neuropsychology, and has been shown to have good psychometric properties \parencite[for review see][]{britesDevelopmentApplicationsSWAN2015}. 
Of interest in this thesis was the subscale inattention, as impaired attention has been reported as one of the most ubiquitous clinical symptoms \parencite{mirskyNosologyDisordersAttention2001} and plays an important role in maintaining relevant information \parencite{chunVisualWorkingMemory2011}. Therefore, the scores of the inattention subscale were included in the analysis (i.e., raw scores). 

\subsubsection{WISC-V}
Scores from the fifth edition Wechsler intelligence scale for children \parencite[WISC-V][]{wechslerWechslerIntelligenceScale2014} were included in the exploratory analysis. More specifically, two primary index scores (i.e., processing speed index and working memory index). For psychometric properties and a more detailed review of this test, see \textcite{wechslerWechslerIntelligenceScale2014} and \textcite{meyerScoresSpaceMultidimensional2018}. Working memory has been stated as an important factor in learning and memory \parencite{cowanWorkingMemoryUnderpins2014} and is prominently associated with P300 in the theoretical framework of context updating \parencite{donchinP300ComponentManifestation1988,lenartowiczUpdatingContextWorking2010}. Therefore, working memory index was included in the exploratory analysis. Processing speed index as second variable was included since it has been shown to be affected in multiple psychiatric disorders and is linked to academic performance \parencite{mayesLearningAttentionWriting2007}.

\subsubsection{Conners 3}
Subscale Learning Problems of the Conners 3 self report \parencite{connersConners2008} was used to explore whether behavioral or neurophysiological measures of learning are associated with learning problems. The psychometric properties of Conners 3 have been reported to be good \parencite{connersConners2008}.

\subsubsection{Child Behavior Checklist}
Five subscales were included from the parent report version of the Child Behavior Checklist questionnaire (CBCL) \parencite{achenbachManualChildBehavior1991}. More specifically, standard scores of the subscales Withdrawn/Depressed, Rule Breaking Behavior, Attention Problems, Anxious/Depressed, and Aggressive Behavior were included. Following an approach done in previous studies, the latter three subscales (i.e., Attention Problems, Anxious/Depressed, Aggressive Behavior) were summed together to arrive at an index for deficits in emotional regulation: CBCL-AAA profile \parencite{biedermanSeverityAggressionAnxietydepression2012,donfrancescoDeficientEmotionalSelfRegulation2015}. Emotional dysregulation has been stated as a main transdiagnostic construct affecting multiple disorders \parencite{sloanEmotionRegulationTransdiagnostic2017} and was therefore included to explore its effect on knowledge index and P300 amplitude decrease.

\subsubsection{Social Responsiveness Scale}
The subscales Social Motivation Problems and Social Cognitions of the Social Responsiveness Scale \parencite{wighamReliabilityValiditySocial2012} were included in the exploratory analysis. The validity and reliability of this questionnaire have been reported to be good \parencite{wighamReliabilityValiditySocial2012}. Including these variables followed the notion that, especially in school settings, social aspects can influence learning and performance \parencite{wentzelAcademicSocialMotivational1998a}.

\subsubsection{Network Analysis}
Altered P300 has been associated with many psychiatric disorders \parencite{polichClinicalApplicationP3002004,surEventrelatedPotentialOverview2009} and seems to reflect a general vulnerability \parencite{surEventrelatedPotentialOverview2009} rather than related to a particular disorder \parencite{duncanEventrelatedPotentialsClinical2009}. Often the findings of altered P300 are mixed when investigating a single disorder \parencite{salgariEventrelatedPotentialsRare2021}. Additionally, P300 has been shown to be sensitive to cognitive dysfunction \parencite{polichCognitiveBiologicalDeterminants1995,potterAssessmentMildHead1999} which itself has been discussed as a transdiagnostic phenomenon in psychopathology moderated by factors such as motivation or emotion \parencite{abramovitchFactorCognitiveDysfunction2021}. This suggests that it may be of importance to study P300 in a transdiagnostic manner. 
Recently, network analysis has been shown to be a promising tool in studying transdiagnostic phenomena \parencite[e.g.,][]{astleAnnualResearchReview2022,borsboomReflectionsEmergingNew2022,chavez-baldiniRelationshipCognitiveFunctioning2021}. 

To explore the question whether the selected transdiagnostic variables provide information on behavioral (i.e., Knowledge Index) and neurophysiological (i.e., P300 decrease) measures obtained from the visual sequence learning paradigm, a network model was estimated. 
As described in \textcite{isvoranuNetworkPsychometricsGuide2022}, network models are statistical structures 
with nodes (i.e., variables) and connecting edges (i.e., estimated statistical relations). Most often these statistical relations are estimated with partial correlations in an undirected manner using Pairwise Markov Random Fields models (PMRFs). In PMRFs the absence of a connection (i.e., edge) between two nodes indicates that these variables are conditional independent after controlling for all other variables (i.e., parietal correlation). The presence of an edge reflects a conditional dependency, with the partial correlation coefficient being the strength of this relationship. A subclass of PMRF models used in this analysis is the Gaussian graphical model \parencite[GGM,][]{lauritzenGraphicalModels1996} which is used when the variables in the network (i.e., nodes) are continuous.
With PMRFs, the predictive relationships between variables in a network model are visualized and would be comparable to computing many exploratory regression analyses \parencite{isvoranuNetworkPsychometricsGuide2022}.  

\paragraph{Procedure}
In a first step, selected variables were plotted to inspect their distribution (see figure \ref{fig:distall} in \nameref{SuppM}). Three highly skewed variables (i.e., subscales from CBCL) were transformed with a nonparanormal transformation \parencite{liuNonparanormalSemiparametricEstimation2009} using the function $huge.npn$ from the huge package \parencite{zhaoHugePackageHighdimensional2012}. This transformation approach uses the cumulative distribution of the data and maps each point to a z-value of a standard normal distribution via the cumulative distribution of the standard normal distribution \parencite{liuNonparanormalSemiparametricEstimation2009}. The estimate of the model was made using the $estimateNetwork$ function of the bootnet package \parencite{epskampPackageBootnet2015}. There are different estimation (and model selection) methods that come with the bootnet package, and the $ggmModSelect$ method was chosen for this analysis. Different estimation methods can produce different results. At this point, however, no clear method is to be preferred and one has to consider the research question and sample size while choosing the estimation method \parencite{isvoranuWhichEstimationMethod2021}. Although it has been observed through simulation studies that with a medium sample size (e.g., N = 1000), $ggmModSelect$ produces estimates with good specificity (i.e., proportion of true absent edges selected in the estimated network) and sensitivity (i.e., proportion of true edges selected in the estimated network) in detecting edges that connect different domains in the network structure (i.e., bridge edges).  
The algorithmic approach of $ggmModSelect$ is described in \textcite{isvoranuNetworkPsychometricsGuide2022}. First, to limit the search space for the model, $ggmModSelect$ estimates various partial correlation matrices with different regularization parameters \footnote{optimizing likelihood function while penalizing for model complexity to a varying degree \parencite{isvoranuNetworkPsychometricsGuide2022}} using glasso \parencite{friedmanSparseInverseCovariance2008}. Next, the resulting models are re-estimated without using regularization and the model structure with the lowest Bayesian information criterion (BIC) is selected. Edges in the selected model are included or removed in an iterative process to once again minizime BIC \parencite{isvoranuNetworkPsychometricsGuide2022}.   
The final model consisted of a partial correlation matrix that was then visualized using the qgraph package \parencite{epskampQgraphNetworkVisualizations2012}. Lastly, the accuracy of edges were estimated using bootstrapped confidence intervals. This was accomplished with non-parametric bootstrapping (bootnet package) in which observations are resampled with replacement and results in 1000 new data sets. When using unregularized estimates of edges (e.g., ggmModSelect), the bootstrapped confidence intervals are reported to be valid as null hypothesis tests \parencite{epskampEstimatingPsychologicalNetworks2018}.
