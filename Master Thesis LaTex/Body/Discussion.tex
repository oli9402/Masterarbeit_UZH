\section{Discussion}\label{Discussion}
In this thesis, a previously found association between the expectancy driven P300 amplitude and successful learning \parencite[e.g.,][]{steinemannTrackingNeuralCorrelates2016,strzelczykNeurophysiologicalMarkersSuccessful2022} could be extended to a sample of children and adolescents with various psychiatric disorders and learning problems. More specifically, behavioral performance (i.e., knowledge index: KI) increased over repetitions, and age explained differences in the first repetition, as well as differences over the course of subsequent repetitions. That is, older participants had higher KI in the first repetition and increased it more strongly over subsequent repetitions. These age effects could represent differences in working memory capacity and would be consistent with previous research showing visual working memory capacity to increase with age during childhood and adolescents \parencite{cowanCapacityAttentionIts2005}. Furthermore, ongoing improvements in executive functions and processing efficiency during adolescents due to brain maturation and myelination of nerve fibers may play a role \parencite{andersonDevelopmentExecutiveFunctions2001}.  
Successful learning can be seen as the formation of cognitive schemas to store and organize multiple items from working memory, thus WM capacity limitations can be overcome to a certain degree and performance can increase when incorporating these schemas during tasks \parencite{bengsonEffectsStrategyVisual2016,cowanMagicalMysteryFour2010,paasCognitiveLoadTheory2014,vanmerrienboerCognitiveLoadTheory2010}. P300 amplitude has been theorized to reflect processes of updating such schemas with surprise (or expectancy) to schema incongruent stimuli \parencite{donchinSurpriseSurprise1981}. Thus, P300 amplitude changes were assessed during learning. In doing so, it was shown that the mean P300 amplitude decreases over the course of learning as the learned sequence becomes increasingly anticipated. Additionally, the decrease in mean P300 amplitude was shown to be stronger for participants with a higher age, which is, as mentioned above, accompanied by increased knowledge of the sequence (i.e., KI) over sequence repetitions. Still, the possibility existed whether this observed decrease in amplitude was due to habituation effects. Previous studies have shown that the amplitude of P300 decreases with repetition in an active oddball discrimination task in young adults for both visual and auditory stimuli modality \parencite{romeroP300Habituation1996}. The authors mentioned that habituation effects can differ depending on task characteristics and physical properties of stimuli used \parencite{romeroP300Habituation1996}. For example, \textcite{courchesneChangesP3Waves1978} showed that P300 amplitude does not habituate for target stimuli (i.e., letter A or B) which require covert response (i.e., counting) whereas stimuli that can be ignored are associated with habituation. In the context of sequence learning, P300 amplitude decreases are expected, but only meaningful if accompanied with an increase in behavioral performance. Therefore, in a subsequent analysis with the aim of strengthening the link between P300 amplitude and knowledge index (KI), it was investigated whether KI can be predicted by mean P300 amplitudes. The results showed a significant association between increased KI and reduced mean P300 amplitude within an individual. Additionally, in an attempt to disentangle P300 amplitude decreases related to successful learning from decreases occurring from habituation or simply ignoring stimuli without learning them, differences in learning categories were investigated regardless of sequence repetitions. 
Using the visual sequence learning paradigm and having the same sequence repeated allowed for finer differentiating of the learning status of spatial locations. Rather than classifying spatial locations during encoding based on later remembered vs. forgotten, they were categorized into learning categories based on how often they were responded to in a correct manner: from unknown over newly learned to fully committed to memory (e.g., known). Investigating the P300 amplitudes of these learning categories revealed that although unknown and newly learned did not differ significantly, once a spatial location was correctly responded to multiple times (i.e., known) the corresponding P300 amplitudes during encoding were significantly lower compared to newly learned and unknown categories. A model including an interaction between learning categories and P300 amplitude did not fit the data better suggesting that differences between learning categories were similar across age, still overall higher age was generally associated with lower P300 amplitudes. Overall, the finding that higher age is associated with lower P300 amplitudes could result from older participants having increased knowledge of the sequence and hence having higher expectancies of upcoming spatial locations. Although it has been observed that in visual paradigms P300 amplitudes decrease with increasing age during childhood and adolescents \parencite{rigginsP300DevelopmentInfancy2020,stigeDevelopmentVisualP3a2007} which may need to be considered when interpreting the found age differences in amplitude. Interestingly though, age differences did not significantly explain P300 amplitude differences in the first repetition, rather it was the decrease in amplitude over repetitions that was influenced by age. Potential explanation could be that in the first repetition, for all participants, the stimuli are equally surprising and therefore no differences in P300 amplitudes are expected. Considering that P300 amplitudes are suggested to be lower in older participants, a second possible explanation could be that these differences in amplitude are mitigated because older participants update their internal model to a larger extent in the first repetition (i.e., higher KI). 


Network analysis can be used as a tool to explore associations in the data and derive new hypotheses from the associations found between nodes in the network \parencite{isvoranuNetworkPsychometricsGuide2022}. Exploring whether differences in mean knowledge index and P300 amplitude decrease can be explained as a network of interacting transdiagnostic constructs was the aim of the second part of this thesis. Focusing more on symptom level rather than disorder labels was motivated by the following points. Firstly, P300 amplitudes have been found to be altered in many psychiatric disorders, but in an unspecific fashion \parencite{duncanEventrelatedPotentialsClinical2009} and are sometimes referred to as reflecting general vulnerability \parencite[e.g.,][]{patrickP300AmplitudeIndicator2006}. Thus, symptoms shared by disorders (e.g., transdiagnostic) may provide insight into the causes of this alteration in P300 amplitudes. Second, the heterogeneity within diagnostic labels can limit conclusions while investigating P300 alterations \parencite[e.g.,][]{kaiserEarlierLaterCognitive2020}. 
The resulting network suggested that how well a participant performs overall in the VSLP (i.e., Mean Knowledge Index) provides information on P300 amplitude decrease over repetitions (i.e., P300 Beta) after controlling for all other included variables in the network. This finding is consistent with established results from the conceptual replication part in the sense that the P300 amplitude is expectancy driven and with greater knowledge of the learned sequence, expectancy decreases. Still, while considering the bootstrapped confidence interval, this association does not seem to be significant. 
\subsection{Limitations}
\subsubsection{P300 Peak Extraction}
Using group average to extract P300 peak latency, as was done in this thesis, has multiple advantages. For example, through averaging trials of many individuals, the signal-to-noise ratio decreases, and the resulting ERP waveform likely includes less random noise, making the peak better identifiable \parencite{luckIntroductionEventrelatedPotential2014}. However, averaging between individuals can be problematic, especially in the investigated age range, where the peak latencies of P300 are expected to differ \parencite{vandinterenP300DevelopmentLifespan2014}. This variation (i.e., latency jitter) can influence peak amplitude \parencite{luckIntroductionEventrelatedPotential2014}. To mitigate this problem, individuals were grouped into age bins of two years. Using a two-year bin size is not uncommon but it has been noted that bin size can affect measurements of ERP (i.e., smaller amplitudes, increased latency, diffuse topographic distribution) due to increased variability between subjects \parencite{rigginsP300DevelopmentInfancy2020}. It is possible that even in narrow age bins as two years, latency variability may differ as a function of age. For example, in a longitudinal study that examined children from 6 to 8 years of age, \textcite{dupuisImplicationsOngoingNeural2015} showed that the intra-individual latency jitter of ERP component (i.e., ERN) decreased in this development time window. Furthermore, considering that P300 peak latency can be modulated by other factors such as gender, pubertal development, intelligence, memory capacity and psychiatric disorders, grouping individuals by age may result in a time window that doesn't capture the P300 component of each individual equally well \parencite{brumbackEfficiencyRespondingUnexpected2012,hansenneP300CognitiveEventrelated2000,lazzaroSingleTrialVariability1997,polichUpdatingP300Integrative2007,rigginsP300DevelopmentInfancy2020}. It is worth noting that a method for calculating individual P300 peak amplitude and latency \parencite[see.,][]{liesefeldEstimatingTimingCognitive2018} failed, possibly due to the low number of trials in each individual. Single trial analysis in children has been described as challenging in previous studies \parencite[e.g.,][]{miyazakiCharacteristicsAuditoryP3001994} possibly due to difficulties in discriminating the P300 signal from background noise \parencite{rigginsP300DevelopmentInfancy2020}. 

Experiments with children and adolescents are marked by multiple challenges resulting from development differences in that age range. These include not only brain maturation that should be considered when choosing a time window, but also tolerance for experimental procedures and, therefore, the paradigm should be child friendly and only include as many trials as necessary \parencite{brookerConductingEventRelatedPotential2020}. Finding a balance between child friendly number of trials and the possibility to extract individual peak amplitude should be considered in future research. 
\subsubsection{Task Difficulty}
In the investigated age range, natural differences in cognitive performance due to brain maturation are expected to occur. Having the same task for every individual regardless of age would have been problematic, thus a shorter and longer version of the VSLP was implemented where participants under the age of nine years completed the shorter version and older participants completed the longer version. This cutoff of nine years followed from a pilot study in which floor effects were found when implementing the longer version of VSLP for all participants \parencite{langerResourceAssessingInformation2017}. It is possible that older participants that are close to the cutoff may experience the longer version of the task more demanding. The effects of higher task demand on P300 amplitude in VSLP are unclear. However, higher task demands have been shown to affect P300 amplitudes in dual-tasks \parencite{isrealP300TrackingDifficulty1980,wickensPerformanceConcurrentTasks1983}. 

\subsubsection{Linking P300 Amplitude with Behavioral Measures}
In this thesis, the P300 amplitude was linked with behavioral measures through learning categories (i.e., Unknown, Newly Learned, Known, Forgotten). This linkage is important since, as already mentioned, P300 amplitudes can be modulated by several factors, thus confounding P300 amplitude changes over repetitions (e.g., habituation). Although the ignored and attended stimuli differ in elicited P300 amplitude, it was shown that P300 amplitude in an ignore condition decreases with repetitions in the same manner as when stimuli are attended to \parencite{beckerDirectingAttentionStimuli1980}. Therefore, even ignoring the learning task could possibly lead to a decrease over repetitions. The rational for linking learning categories with P300 amplitudes was that these are relatively independent of sequence repetition. It could be assumed that this is not entirely the case since learning categories depend on sequence repetition insofar as known categories during learning should tend to increase in later repetitions (see table \ref{tab:RepLC} in \nameref{SuppM}). Additionally, stimuli are categorized as known if responded to correctly at least twice in a row. Thus, no known categories are given in the first repetition. Hence, there is the possibility that habituation contributes to the decrease in P300 amplitude found in known compared to unknown and newly learned categories. Whether this possibility occurred and how much of the effect can be ascribed to habituation wasn't answered in this thesis. Especially since no control condition was used in the paradigm. However, \textcite{steinemannTrackingNeuralCorrelates2016} previously provided evidence against habituation due to time spent on task (in VSLP) by comparing sequence learning blocks with passive sequences blocks that acted as a control condition.   

\subsubsection{Predicting Learning Success Across Participants}
By fitting an exponential to the learning curve, \textcite{steinemannTrackingNeuralCorrelates2016} and \textcite{strzelczykNeurophysiologicalMarkersSuccessful2022} could show that P300 amplitude can predict fast learners versus slow learners. In this thesis, the same approach to predict fast versus slow learners was attempted, but failed. Given the small numbers of sequence repetitions, fitting an exponential function (or functions of different order) was problematic.
Although the aggregated learning curves showed a gradual increase over repetitions, the same must not be true for individual learning curves \parencite{estesProblemInferenceCurves1956,smithSmallBeautifulDefense2018}. How 
excatly individuals learned over repetitions is difficult to interpret. For example, LME estimated negative slopes for repetition in 868 participants. It is possible that many learned in the first repetition, but fail to maintain their knowledge over repetitions. Forgotten learning categories were on average 9.73 \% per participant, which also indicates unstable memory or other contributing factors such as, for example, attention lapses or other problems in executive functions. 
Future research should explore this further, as differentiating between participants is important for the clinical utility of P300 amplitude. 


\subsection{Conclusion and Future Research}
Previously found association between P300 amplitude and successful learning could be extended to a sample of children and adolescents with various psychiatric disorders. Therefore, highlighting its potential as a neurophysiological marker of learning. Such a marker has the possibility of identifying individuals struggling with learning, especially in cases where overt responses are difficult to measure. Second, P300 could be helpful in evaluating interventions aimed at facilitating learning through the monitoring of cognitive processes.  

Future studies could consider joint modeling approach in which brain and behavior measures are modeled simultaneously \parencite{turnerConstrainingCognitiveAbstractions2015}. This approach provides the possibility of bridging the two different levels of analysis by formal modeling cognitive processes and relating the resulting latent parameter to neural measures \parencite{turnerApproachesAnalysisModelbased2017a}. Lately, it was suggested to have a stronger focus on large trial numbers and analysis at the individual participant level rather than solely focusing on large N and group level analysis \parencite{smithSmallBeautifulDefense2018}. Especially with the challenges faced in this thesis, future studies may benefit from including more trials in a way that is still child friendly. 



