\section*{Abstract}
\label{sec:abs}
\addcontentsline{toc}{section}{\nameref{sec:abs}}


% CONTENT OF ABS HERE--------------------------------------
Memory and learning are very important in everyday life. In individuals with psychiatric disorders, these cognitive abilities are often compromised. Especially in childhood and adolescents, such deficits can negatively impact many outcomes later in life such as, for example, academic success or health. Identifying learning problems can be challenging in cases where individuals struggle to provide behavioral responses. Furthermore, behavioral responses provide little information about underlying cognitive processes that are relevant to learning and memory. Event-related potentials (ERP), on the other hand, provide a near real-time measure of brain activity occurring in learning settings. Previous studies have linked an ERP component (P300) with successful learning using a visual sequence learning paradigm (VSLP) that allows tracking gradual memory formation on a behavioral and neurophysiological level. Through learning in VSLP, subjective expectancy of upcoming stimuli is systematically modulated, which in turn is suggested to influence P300 amplitude.
This thesis extended the previously established link between P300 amplitude and successful learning to a sample of children and adolescents with various psychiatric disorders. More specifically, P300 amplitude was shown to predict learning success within subjects. Furthermore, P300 amplitude decreased over the course of learning regardless of sequence repetition. Associations between the degree of decrease in P300 amplitude and learning-related transdiagnostic constructs were explored through a network analysis. The network yielded no direct associations with the degree of decrease in the amplitude of P300. 
In summary, these results highlight P300 as a neurophysiological marker of learning and its potential clinical value. 
\newpage

